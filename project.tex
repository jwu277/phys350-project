%% Standard start of a latex document
\documentclass[letterpaper,12pt]{article}
%% Always use 12pt - it is much easier to read
%% Things written after '%' sign, are ignored by the latex editor - they are how to introduce comments into your .tex source
%% Anything mathematics related should be put in between '$' signs.

%% Set some names and numbers here so we can use them below
\newcommand{\name}{James Wu} %%%%%%%%%%%%%%% ---------> Change this to your name
\newcommand{\studentnumber}{92277235} %%%%%%%%%%%%%%% ---------> Change this to your student number

%%%%%%
%% There is a bit of stuff below which you should not have to change
%%%%%%

%% AMS mathematics packages - they contain many useful fonts and symbols.
\usepackage{amsmath, amsfonts, amssymb, bm}

%% The geometry package changes the margins to use more of the page, I suggest
%% using it because standard latex margins are chosen for articles and letters,
%% not homework.
\usepackage[paper=letterpaper,left=25mm,right=25mm,top=30mm,bottom=30mm]{geometry}
%% For details of how this package work, google the ``latex geometry documentation''.

%% Fancy headers and footers - make the document look nice
\usepackage{fancyhdr} %% for details on how this work, search-engine ``fancyhdr documentation''
\pagestyle{fancy}

\usepackage{graphicx}

\setlength{\headheight}{15pt}

%% The header
\chead{PHYS 350: Speed of Sound in a Solid} % homework number in top-centre

%% The footer
\cfoot{Page \thepage} % page in middle

%% These put horizontal lines between the main text and header and footer.
\renewcommand{\headrulewidth}{0.4pt}
\renewcommand{\footrulewidth}{0.4pt}
%%%

%%%%%%
%% The above stuff is the same as the first template, but now we are starting to prove things, so we'd like to have a
%% good proof environment that gives us nice formatting and a little square at the end.
%% We'd also like a nice Result environment that prints that up nicely too.
%% Thankfully this exists in latex in the amsthm package
\usepackage{amsthm}
\newtheorem*{thm}{Theorem}
%% This creates a new theorem-like environment called "result", that will be titled "Result".
%% See below for examples of how to use this.
%%%%%%
\usepackage{enumitem}
%% This package allows us to make nice ordered lists with numbers, letters or roman numerals

\usepackage{titlesec}
\titlespacing*{\subsection}{0pt}{0pt}{3.0ex}
\titlespacing*{\subsubsection}{0pt}{3.0ex}{0.5ex}

\usepackage[hang,flushmargin]{footmisc}

\setlength{\parindent}{0em}
\setlength{\parskip}{0.5em}

\allowdisplaybreaks

\usepackage{empheq}

\usepackage{sectsty}
\usepackage{subcaption}

\newcommand*\wfbox[1]{\fbox{\hspace{0.4em}#1\hspace{0.4em}}}

%% Useful commands
\renewcommand*{\qed}{\hfill\ensuremath{\square}}

\newcommand*{\uvec}[1]{\hat{\bm{#1}}}

\newcommand*{\deriv}[2]{\frac{d #1}{d #2}}
\newcommand*{\pderiv}[2]{\frac{\partial #1}{\partial #2}}
\newcommand*{\nderiv}[3]{\frac{d^{#3} #1}{d #2^{#3}}}
\newcommand*{\npderiv}[3]{\frac{\partial^{#3} #1}{\partial #2^{#3}}}
\newcommand*{\divg}[1]{\nabla \cdot \mathbf{#1}}

\newcommand*{\abs}[1]{\left| #1 \right|}
\newcommand*{\norm}[1]{\abs{\abs{\mathbf{#1}}}}

\newcommand*{\ev}[1]{\left<#1\right>}

\renewcommand*{\Re}[1]{\text{Re}\left(#1\right)}
\renewcommand*{\Im}[1]{\text{Im}\left(#1\right)}

\newcommand*{\qimg}[2]{\\ \begin{center}\includegraphics[scale=#1]{#2}\end{center}}

\newcommand*{\Arg}[1]{\text{Arg}\left(#1\right)}

\newcommand*{\Log}[1]{\text{Log}\left(#1\right)}

%%

\begin{document}

\begin{center}
    \subsection*{Speed of Sound in a Solid with Lagrangian Mechanics}
    James Wu, \quad 92277235
\end{center}

\begin{flushleft}

    \subsubsection*{Introduction}
    In this problem we derive an expression for the longitudinal speed of sound in a homogeneous isotropic elastic solid by modelling the solid as a three-dimensional cubic lattice. We first determine the speed of sound in terms of microscopic parameters of the material; subsequently we shall express the speed of sound in macroscopic material properties. For simplicity, we consider the case of a rectangular prism. Note that the oscillations of atoms about equilibrium will be small compared to the bond length (this is not a small oscillations problem, however).

    \subsubsection*{Part I}
    Consider a solid rectangular prism of dimensions $a \times b \times c$ at rest. We tie our inertial reference frame to a corner of the prism with axes along those of the prism as shown in \textbf{Figure~\ref{fig:I1}} (so more precisely, we require this corner to be at rest).
    \begin{figure}[h]
        \centering
        \includegraphics[scale=0.8]{images/i1.jpg}
        \caption{Sketch of the solid prism and our inertial reference frame for \textbf{Part I}. The solid is of side lengths $a$, $b$, and $c$ along the $x$-, $y$-, and $z$-axes, respectively.}
        \label{fig:I1}
    \end{figure}
    We may model this solid prism as a $M \times N \times P$ cubic lattice of point masses, each with mass $m$. These point masses represent the solid's atoms. This is illustrated in \textbf{Figure~\ref{fig:I2}}.
    \begin{figure}[h]
        \centering
        \begin{subfigure}[b]{0.6\textwidth}
            \includegraphics[width=\textwidth]{images/I2.jpg}
            \caption{}
            \label{fig:I2}
        \end{subfigure}
        \quad
        \begin{subfigure}[b]{0.3\textwidth}
            \includegraphics[width=\textwidth]{images/I3.jpg}
            \caption{}
            \label{fig:I3}
        \end{subfigure}
        \caption{(a) Sketch of the crystal lattice projected on the $xy$-plane. The black dots represent atoms; each atom in the lattice is spaced out by a distance $l_0$ at equilibrium. In total there are $M$, $N$, and $P$ atoms along the $x$-, $y$-, and $z$-axes, respectively. Although this is not depicted, this lattice repeats similarly along the $z$-axis. (b) Sketch of the interaction between an interior atom (square) and its six neighbours (circles). This interaction is modelled as a spring of equilibrium length $l_0$ and spring constant $K$. It follows that these springs lie along the $x$-, $y$-, and $z$-axes (two on each).}
    \end{figure}
    The bond potential energy between two adjacent atoms experiences a local minimum at an equilibrium length $l_0$. Applying the small oscillations approximation to this stable equilibrium, we model the interaction between two adjacent atoms as a harmonic oscillator (i.e. spring) of equilibrium displacement $l_0$ and spring constant $K$. Furthermore, we take the interaction between non-adjacent atoms to be negligible. For interior (i.e. not on the solid's boundary) atoms, this is depicted in \textbf{Figure~\ref{fig:I3}} (as we shall see, the boundary atoms and their interactions with the external environment determine the boundary conditions). Finally, we neglect gravity throughout this problem.\newline\newline
    \textbf{As a bonus challenge, complete all parts of this problem for an arbitrary prismatic solid. That is, take} $P$ \textbf{to be a function of} $j$\textbf{,} $P(j)$\textbf{. As a second bonus, relax the assumption of isotropy. That is,} $l_0$ \textbf{and} $K$ \textbf{need not be equal for springs along different coordinate axes.}\newline\newline
    To get started, how many degrees of freedom in this system? What set of generalized coordinates could you use to describe this system?

    \subsubsection*{Part II}
    We may label each atom in the lattice with indices $i$, $j$, and $k$. We shall label the corner atom at the origin with $i = 0$, $j = 0$, and $k = 0$. Then our indices range over $0 \leq i \leq M-1$, $0 \leq j \leq N-1$, and $0 \leq k \leq P-1$. The atom adjacent in the $x$-direction to the origin corner atom, for example, would have indices $(i, j, k) = (1, 0, 0)$. Two atoms above this atom in the $z$-direction would have indices $(i, j, k) = (1, 0, 2)$, etc. We may denote the (global i.e. when every atom is at equilibrium) equilibrium $x$, $y$, and $z$ positions of the $(i, j, k)$ atom as $\tilde{x}_{ijk}$, $\tilde{y}_{ijk}$, and $\tilde{z}_{ijk}$, respectively. \newline\newline
    Determine $\tilde{x}_{ijk}$, $\tilde{y}_{ijk}$, and $\tilde{z}_{ijk}$ in terms of $i$, $j$, $k$, and $l_0$. What are the dimensions $a$, $b$, and $c$ of the prism in terms of $M$, $N$, $P$, and $l_0$?

    \subsubsection*{Part III}
    We similarly may denote the (current) $x$, $y$, and $z$ positions of the $(i, j, k)$ particle as $x_{ijk}$, $y_{ijk}$, and $z_{ijk}$, respectively. In terms of these coordinates, find the total kinetic and potential energies of the system. What is the Lagrangian? Are there any conserved quantities? Find them. Neglect gravitational effects.

    \subsubsection*{Part IV}
    Find the Euler-Lagrange equations of the system (note that the equations for boundary atoms will be different in pattern to those for interior atoms).

    \subsubsection*{Part V}
    Now let us consider longitudinal acoustic wave (sound) propagation along the $x$-axis. We do so by taking $y_{ijk}(0) = \tilde{y}_{ijk}$, $z_{ijk}(0) = \tilde{z}_{ijk}$, and $\dot{y}_{ijk}(0) = \dot{z}_{ijk}(0) = 0$ in our initial conditions. Furthermore, we consider the case where the acoustic waves are symmetrical across the cross-section (to the $x$-axis). We do so by requiring $x_{ijk}$ to be invariant across $j$ and $k$ in the initial conditions (when we add boundary conditions later, those too will be symmetrical). $y_{ijk}$ and $z_{ijk}$ will remain constant as a result. Furthermore, $x_{ijk}$ will remain invariant across $j$ and $k$. The longitudinal wave propagation is therefore described by the longitudinal displacement (from equilibrium) $X_i$ of point $(i,j,k)$: $X_i = x_i - \tilde{x}_i$. We only need the $i$ index given that $x_{ijk}$ only depends on $i$. As for $\tilde{x}_{ijk}$, you may verify that your expression in \textbf{Part II} is independent of $j$ and $k$. \newline\newline
    Show that $\ddot{y}_{ijk}$ and $\ddot{z}_{ijk}$ are initially zero from their EL equations under our assumptions for the initial condition. Since $\dot{y}_{ijk}$ and $\dot{z}_{ijk}$ are initially zero, $y_{ijk}$ and $z_{ijk}$ will consequently remain at their equilibrium values. Simplify the EL equations for $x$, applying this fact. Verify that $\ddot{x}_{ijk}$ is independent of $j$ and $k$ at $t=0$. Since $x_{ijk}$ and $\dot{x}_{ijk}$ are invariant with respect to $j$ and $k$ at $t=0$, we have that $x_{ijk}$ will forever be independent of $j$ and $k$ (notice that the EL equations are time invariant). With all this in mind, we may reduce the $3(MNP - 1)$ equations to $M - 1$ equations ($x_0$ is fixed). Rewrite these equations in terms of $X_i$.

    \subsubsection*{Part VI}
    Turning now to the theory of finite differences, we may approximate the derivatives of a twice continuously differentiable function $f(x)$ in a mesh as
    $$\deriv{f}{x}\bigg|_{x=x_i} \approx \frac{f(x_{i+1}) - f(x_i)}{\Delta x}, \; \nderiv{f}{x}{2}\bigg|_{x=x_i} \approx \frac{f(x_{i+1}) + f(x_{i-1}) - 2f(x_i)}{\left(\Delta x\right)^2}$$
    for a given mesh size of $\Delta x$. We (you) will make use of the first approximation later. Of course, these finite differences have analogous formulae for partial derivatives.\newline\newline
    Let us now concern ourselves with somehow differentiating $X_{ijk}$. As it stands, the equations appear to be finite difference equations, however $X_{ijk}$ this is not a function of $x$, $y$, and $z$. However, $X_{ijk}(t)$ is a discretization of some twice differentiable function $\tilde{X}(x, y, z, t)$, where the EL equations serve as the difference equations. We may interpret $X_{ijk}$ to be $\tilde{X}$ evaluated at the point where the $(i, j, k)$ atom's equilibrium location is. Because there are a very large number of atoms along each dimension in a macroscopic solid, the finite difference discretization is very accurate; that is, the values of $X_{ijk}$ will be approximately equal the actual values of $\tilde{X}$ (which is governed by the corresponding differential equation) evaluated at those points.\newline\newline
    Express the equations for the interior points obtained in \textbf{Part V} as a set of finite difference equations for $\tilde{X}$. What should the mesh spacing $\Delta x$ be?

    \subsubsection*{Part VII}
    As one of the most important equations in physics, the wave equation is
    $$\npderiv{f}{t}{2} = c^2 \nabla^2 f$$
    This represents the propagation of a quantity $f$ through space. The constant $c$ is known as the \textit{wave speed}; this represents the speed at which $f$ propagates through space. Notice that the set of equations obtained in \textbf{Part VI} is a \textit{pseudo}-discretization of the wave equation for $\tilde{X}$. \newline\newline
    ``Undiscretize'' these equations to recover the wave equation for $\tilde{X}$. Longitudinal sound wave in a solid are hence carried through some continuous (in fact continuously twice differentiable) field $\tilde{X}$; at lattice points, this field is simply the longitudinal displacement of the corresponding atom. Now find the wave speed $c$ in terms of the quantities we have encountered thus far. The equations in \textbf{Part VI} represent a \textit{pseudo}-discretization of the wave equation; in fact, this is better than a typical finite difference discretization. Why is this the case (recall that finite difference discretizations are used to numerically compute the solutions to differential equations)?

    \subsubsection*{Part VIII}
    Let us take a detour now to the mechanics of materials. Consider a uniaxial deformation of the prism where the $yz$-face opposite to the origin is displaced by an amount $\delta$. In terms of generalized coordinates, we now constrain $x_{Mjk} = a + \delta$ for all $0 \leq j \leq N-1, 0 \leq k \leq P-1$ under the deformation. If initially the system was at rest with no deformation applied (i.e. with no constraint on $x_{Mjk}$), and we take the system after the displacement to be at rest (this could be in steady state, where the oscillations of the atoms have been damped; for sound propagation we have neglected damping), then the \textit{elastic modulus} $E$ ($E_x$ if you are not assuming isotropy) is defined to be the constant of proportionality such that the increase in energy per total (initial) volume is $\frac{1}{2}E\varepsilon^2$, where $\varepsilon = \frac{\delta}{a}$ is known as the \textit{strain} of the deformation (the elastic modulus is more commonly defined in terms of forces; in the context of Lagrangian mechanics, however, we shall continue to formulate phenomena in terms of generalized coordinates and energies). Finally, the density $\rho$ of a material is its mass per unit volume.\newline\newline
    Express the speed of sound $c$ in the solid in terms of its macroscopic properties.

    \subsubsection*{Part IX}
    We have finally found the speed of longitudinal sound waves in a homogeneous isotropic elastic solid, albeit for the case of a rectangular prism. Our results, however, apply to any such prismatic solid. To finish off, as a bonus exercise let us consider the boundary conditions of the prism. The boundary condition for $\tilde{X}$ at $x = 0$ is already constrained to be
    $$\tilde{X}(x=0, y, z, t) = 0$$
    as we require the corner to be fixed at rest. Symmetry in $y$ and $z$ requires this to be the case for the entire $yz$ face. At the other end's $yz$ face, however, we are free to specify any boundary condition provided that it is symmetric in $y$ and $z$. For example, we may specify the Dirichlet boundary condition
    $$\tilde{X}(x=a, y, z, t) = p$$
    the mixed boundary condition
    $$\pderiv{\tilde{X}}{x}\bigg|_{x=a} = q$$
    or the Robin boundary condition
    $$\alpha\tilde{X}(x=a, y, z, t) + \beta\pderiv{\tilde{X}}{x}\bigg|_{x=a} = \gamma$$
    for $\alpha, \beta \neq 0$.\newline\newline
    Physically speaking, what is happening to the $x = a$ face for the Dirichlet, mixed, and Robin boundary conditions, respectively? How do the constants $p, q, \alpha, \beta, \gamma$ relate to what is being applied to the $x = a$ face for each boundary condition? It may help to create ``ghost'' points $X_{M,j,k}$ and use the right finite difference introduced in \textbf{Part VI} for the first derivative.

\end{flushleft}

\end{document}
